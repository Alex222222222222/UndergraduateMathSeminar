\subsection{Theta Function}

      \begin{frame}
            \frametitle{Theta Function}
            Given a graph $ G $, the $ \theta(G) $ is defined by
            \begin{equation}
                  \theta(G) = \inf_{\{v_1, v_2, \dots, v_n\},c} \max_{i} \frac{1}{\left<c,v_{i}\right>^2}
            \end{equation}
            where $ \{v_1, v_2, \dots, v_n\} $ is an orthonormal representation of $ G $, and c is any unit vector does not orthogonal to $ v_i $.

            \pause

            \begin{lemma}
                  There always exist such an $c$ and orthonormal representation $ \{v_1, v_2, \dots, v_n\} $ such that
                  \begin{equation}
                        \theta(G) = \max_{i} \frac{1}{\left<c,v_{i}\right>^2}
                  \end{equation}
            \end{lemma}

            This could be proved by proving the set of all possible cases of $ \{v_1, v_2, \dots, v_n,c \} $ is compact. And the function $ \max \frac{1}{\left<c,v_{i}\right>^2} $ is continuous.
      \end{frame}

\subsection{Theta Function of Product Graph}

      \begin{frame}
            \frametitle{Theta Function of Product Graph}
            \begin{lemma}
                  Given graph $ G $ and $ H $, then
                  \begin{equation}
                        \theta(G \times H) \leq \theta(G) \theta(H)
                  \end{equation}
            \end{lemma}

            \pause

            \begin{proof}
                  Let $ \{v_1, v_2, \dots, v_n\} $ and $ \{w_1, w_2, \dots, w_m\} $ be orthonormal representation of $ G $ and $ H $ and $ c_{v} $ and $ c_{w} $ such that
                  \begin{equation}
                        \max \left\{ \frac{1}{\left<c_{v},v_{i}\right>^2} : i=1,2,\dots,n \right\} = \theta(G)
                  \end{equation} 
                  and 
                  \begin{equation}
                        \max \left\{ \frac{1}{\left<c_{w},w_{i}\right>^2} : i=1,2,\dots,m \right\} = \theta(H)
                  \end{equation}
            \end{proof}
      \end{frame}

      \begin{frame}
            \begin{proof}
                  Then
                  \begin{eqnarray}
                        \theta(G \times H) &\leq& 
                        \max \left\{ \frac{1}{\left<c_{v} \circ c_{w},v_{i} \circ w_{j}\right>^2} \right\} \\
                        &=& \max \left\{ \frac{1}{\left<c_{v},v_{i}\right>^2 \left<c_{w},w_{j}\right>^2} \right\} \\
                        &=& \max \left\{ \frac{1}{\left<c_{v},v_{i}\right>^2} \right\} \max \left\{ \frac{1}{\left<c_{w},w_{j}\right>^2} \right\} \\
                        &=& \theta(G) \theta(H) \\
                  \end{eqnarray}
            \end{proof}
      \end{frame}

\subsection*{Relation of Theta Function and Alpha Function}

      \begin{frame}
            \frametitle{Relation of Theta Function and Alpha Function}

            \begin{lemma}
                  \begin{equation}
                        \theta(G) \geq \alpha(G)
                  \end{equation}
            \end{lemma}

            \pause

            \begin{proof}
                  Let $ \{1,2,\dots,k\} $ be the set of vertices of $ G $ such that every point is not adjacency in $ G $. And $ k = \alpha(G) $
      
                  Let $ \{v_1, v_2, \dots, v_n\} $ be an orthonormal representation of $ G $ and $ c $ such that
                  \begin{equation}
                        \max \left\{ \frac{1}{\left<c,v_{i}\right>^2} \right\} = \theta(G)
                  \end{equation}
                        
                  Then
                  \begin{eqnarray}
                        1 &=& c^{2}
                        \geq \sum_{i=1}^{k} \left<c,v_{i}\right>^{2}
                        \geq \frac{k}{\theta(G)}
                  \end{eqnarray}
            \end{proof}
      \end{frame}

\subsection*{Relation of Theta Function and Shannon Capacity}
      
      \begin{frame}
            \frametitle{Relation of Theta Function and Shannon Capacity}

            \begin{theorem}
                  Given a graph $ G $, then
                  \begin{equation}
                        \theta(G) \geq \Theta(G)
                  \end{equation}
            \end{theorem}

            \pause

            \begin{proof}
                  % TODO
                  \begin{eqnarray}
                        \Theta(G) &=& \lim_{n} \sqrt[n]{\alpha(G^{n})} \\
                        &\leq& \lim_{n} \sqrt[n]{\theta(G^{n})} \\
                        &\leq& \lim_{n} \sqrt[n]{\theta(G)^{n}} \\
                        &=& \theta(G)
                  \end{eqnarray}
            \end{proof}
      \end{frame}

      \begin{frame}
            The final result we want to prove today,
            \begin{equation}
                  \Theta(C_{5}) \le \theta(C_{5}) \le \sqrt{5}
            \end{equation}
      \end{frame}