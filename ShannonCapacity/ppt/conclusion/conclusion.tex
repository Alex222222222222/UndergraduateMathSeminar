\section{Conclusion and Discussion}

\begin{frame}
      \frametitle{Conclusion}
      \begin{itemize}
            \item We have proved that $Theta(C_{5}) = \sqrt{5}$
            \item We could actually proved that $\Theta(C_{n})$ is equal to $ n\frac{\cos(\pi/n)}{1+\cos(\pi/n)} $.
      \end{itemize}
\end{frame}

\begin{frame}
      \frametitle{Open Questions}
      \begin{itemize}
            \item Although $\Theta(C_{n})$ is equal to $ n\frac{\cos(\pi/n)}{1+\cos(\pi/n)} $, but we still don't know the exact value of $\Theta(C_{n})$. Even for $n=7$
            \item Is there any good lower bound for $\Theta(C_{n})$?
            \item Is there any patterns for $n$ such that $\Theta(c_{n})$ is hard to compute?
      \end{itemize}
\end{frame}

\begin{frame}
      \frametitle{Discussion}

      In the real world cases, we always have some kind of relay between the sender and receiver. So the new channel is kind of composite of two channel. Can we compute Shannon Capacity these channels independently and then combine them together to get the Shannon Capacity of the new Channel?
\end{frame}

\section{Reference}

\begin{frame}
      \frametitle{Reference}
      \begin{itemize}
            \item \href{https://ieeexplore.ieee.org/stamp/stamp.jsp?arnumber=1055985}{On the Shannon Capacity of a Graph} by Laszlo Lovasz
            \item \href{https://ieeexplore.ieee.org/stamp/stamp.jsp?tp=&arnumber=1056798}{The zero error capacity of a noisy channel} by Claude Shannon
      \end{itemize}
\end{frame}